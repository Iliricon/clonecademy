\begin{tabularx}{\textwidth}{|p{.2\textwidth}|X|}
	\hline
	ID & 32 \\
	\hline
	Benutzerrolle & Admin \\
	\hline
	Name & Adminrechte entziehen\\
	\hline
	Beschreibung & Als Admin möchte ich einem Moderator die Adminrechte wieder entziehen können. Wenn ein Admin einen anderen Adminnutzer in der Adminansicht auswählt, gibt es dort einen Button, mit dem man dem Admin seine Adminrechte entziehen kann.  \\
	\hline
	Akzeptanzkriterium & Nach Entzug der Adminrechte kann der betreffende Nutzer keine administrativen Aufgaben mehr erledigen. Er verliehrt Zugriff auf alle Bereiche der Seite und Schaltflächen, die nur für Administratoren sichtbar sind. Wenn er davor Moderator war, behält er die Rechte dieser Gruppe weiterhin. \\
	\hline
	Story Points & 3 \\
	\hline
	Entwickler & Tobias Huber \\
	\hline
	Umgesetzt in Iteration & 16\\
	\hline
	Tatsächlicher Aufwand & 4h\\
	\hline
	Velocity & \\
	\hline
	Bemerkung & Beim Entzug der Adminrechte wird die Verknüpfung zwischen dem betreffenden Nutzer und der Gruppe "Admin" in der Datenbank gelöscht.\\
	\hline
\end{tabularx}
\vspace{20pt}
