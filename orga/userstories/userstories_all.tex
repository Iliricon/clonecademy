\documentclass
[english,accentcolor=tud1c]
{tudreport}

\usepackage[T1]{fontenc}
\usepackage[utf8]{inputenc}
\usepackage{amstext}
\usepackage{amsmath}
\usepackage{graphicx}
\usepackage{setspace}
\usepackage{multicol}
\usepackage{mathtools}
\usepackage{dsfont}
\usepackage{units}
\usepackage{subfigure}
\usepackage{color}
\usepackage{booktabs}
\usepackage{fancyref}
\usepackage[ngerman,english]{babel}
\usepackage{pdfpages}
\usepackage{tabularx}

\title{User Stories\\Clonecademy}
\author{Tobias Huber, Leon Wiedmann, Ilhan , Claas Alexander Voelcker}


\begin{document}

	%=========================================================

	\maketitle

	\chapter{Iteration}

	\begin{tabularx}{\textwidth}{|p{.2\textwidth}|X|}
	\hline
	ID & 2\\
	\hline
	Benutzerrolle & Nutzer\\
	\hline
	Name & Authentifizierung\\
	\hline
	Beschreibung & Als Nutzer möchte ich mich auf der Seite als Nutzer einloggen können.\\
	\hline
	Akzeptanzkriterium & Das Authentifizerungsfenster wird korrekt angezeigt. Bei gültiger Eingabe seiner Daten wird der Nutzer auf die Kursübersicht weitergeleitet und ein Authetifizierungstoken wird generiert, der zur Authetifizierung weitergereicht wird. Bei ungültiger Eingabe wird der Nutzer durch ein Popup darauf hingewiesen und nicht eingelogt.\\
	\hline
	Story Points & 2 \\
	\hline
	Entwickler & Leonhard Wiedmann\\
	\hline
	Umgesetzt in Iteration & 1\\
	\hline
	Tatsächlicher Aufwand & 3.5 h\\
	\hline
	Velocity & 1.75 h/SP\\
	\hline
	Bemerkung & \\
	\hline
\end{tabularx}
\vspace{20pt}
 \\
	\begin{tabularx}{\textwidth}{|p{.2\textwidth}|X|}
	\hline
	ID & 3\\
	\hline
	Benutzerrolle & Moderator*in\\
	\hline
	Name & Neuen Kurs anlegen\\
	\hline
	Beschreibung & Als Moderator*in möchte ich einen neuen Kurs anlege können. Dazu gehe ich auf die Kursübersicht und wähle die Schaltfläche ''Add Course'' aus. Ein Editor erlaubt mir, Name und Kategorie des Kurses auszuwählen. Danach kann ich den Kurs speichern und er taucht auf der Übersichtsseite auf. Der Kurs ist erste einmal nur für Moderator*innen sichtbar.
\\
	\hline
	Akzeptanzkriterium & Der neu angelegte Kurs mit Name und Kategorie wird persistent gespeichert und ist für Moderator*innen einsehbar.\\
	\hline
	Story Points & 3\\
	\hline
	Entwickler & Claas Völcker \& Leonhard Wiedmann\\
	\hline
	Umgesetzt in Iteration & 1\\
	\hline
	Tatsächlicher Aufwand & 3 h\\
	\hline
	Velocity & 1.5 h/SP\\
	\hline
	Bemerkung & Die genaue Funktionsweise des Editors wurde in einem Gespräch mit den Auftraggeber*innen festgelegt.\\
	\hline
\end{tabularx}
\vspace{20pt}



	\chapter{Iteration}

	\begin{tabularx}{\textwidth}{|p{.2\textwidth}|X|}
	\hline
	ID & 5\\
	\hline
	Benutzerrolle & Nutzer*in\\
	\hline
	Name & Kurs auswählen\\
	\hline
	Beschreibung & Als Nutzer*in möchte ich aus einer Kursübersicht einen Kurs auswählen und im Folgenden die Fragen des Kurses beantworten können. Nachdem ich den Kurs ausgewählt habe, soll er in meiner persönlichen Kursübersicht samt Fortschrittsangabe angezeigt werden.\\
	\hline
	Akzeptanzkriterium & Der ausgewählte Kurs wird vollständig angezeigt und die Fragen können bearbeitet und zugleich jederzeit abgebrochen werden.Der Fortschritt wird gespeichert und in einer persönlichen Statistik angezeigt. Nach Abbrechen einer Frage kann die Bearbeitung derselben jederzeit fortgesetzt werden.\\
	\hline
	Story Points & 7\\
	\hline
	Entwickler & Ilhan Simsiki\\
	\hline
	Umgesetzt in Iteration & 3\\
	\hline
	Tatsächlicher Aufwand & 11 h\\
	\hline
	Velocity & 1.57 h/SP\\
	\hline
	Bemerkung & \\
	\hline
\end{tabularx}
\vspace{20pt}
 \\
	\begin{tabularx}{\textwidth}{|p{.2\textwidth}|X|}
		\hline
		ID & 1 \\
		\hline
		Benutzerrolle & Nutzer \\
		\hline
		Name & Nutzer erstellen \\
		\hline
		Beschreibung & Als Nutzer möchte ich auf der Seite einen eigenen Account anlegen können. Mit Ausfüllen des Anmeldeformulars mit den Pflichtfeldern:
		E-Mail-Adresse,
		Benutzername,
		Passwort.
		Und den Optionalen Feldern:
		Vorname,
		Nachname,
		Alter \\
		\hline
		Akzeptanzkriterium & Das Anmeldeformular wird korrekt dargestellt. Mit dem Klicken des ''Registrieren'' Buttons und Ausfüllen der Pflichtfelder wird der Account persistent in der Datenbank gespeichert. \\
		\hline
		Story Points & 2 \\
		\hline
		Entwickler & Tobias Huber \\
		\hline
		Umgesetzt in Iteration & \\ 
		\hline
		Tatsächlicher Aufwand &  \\
		\hline
		Velocity &  \\
		\hline
		Bemerkung &  \\
		\hline
\end{tabularx}
\vspace{20pt}
 \\
	\begin{tabularx}{\textwidth}{|p{.2\textwidth}|X|}
	\hline
	ID & 8\\
	\hline
	Benutzerrolle & Nutzer*in\\
	\hline
	Name & Frage beantworten\\
	\hline
	Beschreibung & Ein*e Nutzer*in möchte eine Frage beantworten. Dabei wird jede Frage ausgewertet, und dem*der Nutzer*in wird angezeigt, ob die Eingabe richtig oder falsch war. War die Beantwortung korrekt, wird die nächste Frage angezeigt.\\
	\hline
	Akzeptanzkriterium & Die Frage kann beantwortet werden und das Ergebnis wird in der Datenbank unter Statistik gespeichert.\\
	\hline
	Story Points & 7 Story Points\\
	\hline
	Entwickler & Leonhard Wiedmann\\
	\hline
	Umgesetzt in Iteration & 4\\
	\hline
	Tatsächlicher Aufwand & 14 h\\
	\hline
	Velocity & 2 h/SP\\
	\hline
	Bemerkung & Es werden neue Daten in die Datenbank geschrieben.\\
	\hline
\end{tabularx}
\vspace{20pt}



	\chapter{Iteration}

	\begin{tabularx}{\textwidth}{|p{.2\textwidth}|X|}
	\hline
	ID & 6\\
	\hline
	Benutzerrolle & Nutzer\\
	\hline
	Name & Statistik einsehen\\
	\hline
	Beschreibung & Als Nutzer möchte ich einen Button ''Meine Statistik'' anklicken können. Nach Anklicken erscheint eine Seite, auf der ich meine bisherigen beantworteten Fragen sehen kann: Wie viele Fragen habe ich richtig beantwortet, wie viele sind falsch.\\
	\hline
	Akzeptanzkriterium & Die Statistik des Nutzers wird nach Klicken auf die dafür vorgesehene Schaltfläche vollständig angezeigt.\\
	\hline
	Story Points & 8 Story Points\\
	\hline
	Entwickler & Claas Völcker\\
	\hline
	Umgesetzt in Iteration & 4\\
	\hline
	Tatsächlicher Aufwand & 8 h\\
	\hline
	Velocity & 1 h/SP\\
	\hline
	Bemerkung &  Es werden keine Daten in der Datenbank geändert.\\
	\hline
\end{tabularx}
\vspace{20pt}
 \\
	\begin{tabularx}{\textwidth}{|p{.2\textwidth}|X|}
	\hline
	ID & 7\\
	\hline
	Benutzerrolle & Moderator\\
	\hline
	Name & Multiple Choice Fragen einpflegen\\
	\hline
	Beschreibung & Als Moderator kann ich einem Kurs neue Fragen hinzufügen. Dazu wähle ich auf der Kursübersicht die Schaltfläche neue Frage anlegen aus. Ich kann daraufhin in einem Interface den Name der Frage, den Text der Frage und mögliche Antwortmöglichkeiten eingeben. In einem weiteren Fenster kann ich einen Text eingeben, der den Nutzenden bei erfolgreicher Bearbeitung angezeigt wird und einen Feedbacktext bei falscher Beantwortung.\\
	\hline
	Akzeptanzkriterium & Eine neu angelegte Frage wird persistent unter dem Kurs gespeichert und ist nach Freigabe für Nutzende bearbeitbar.\\
	\hline
	Story Points & 8 Story Points\\
	\hline
	Entwickler & Leonhard Wiedmann\\
	\hline
	Umgesetzt in Iteration & 4\\
	\hline
	Tatsächlicher Aufwand & 15 h\\
	\hline
	Velocity & 1.875 h/SP\\
	\hline
	Bemerkung & Moderatoren können nur Fragen in ihren eigenen Kursen anlegen. Es werden neue Daten in die Datenbank geschrieben, aber keine existierenden verändert oder gelöscht.\\
	\hline
\end{tabularx}
\vspace{20pt}



	\chapter{Iteration}

	\begin{tabularx}{\textwidth}{|p{.2\textwidth}|X|}
	\hline
	ID & 9\\
	\hline
	Benutzerrolle & Admin\\
	\hline
	Name & Nutzerliste einsehen\\
	\hline
	Beschreibung & Als Admin möchte ich mir eine Übersicht über die vorhandenen Nutzer*innen anzeigen lassen und für alle eine Detailansicht sehen können (die Implementierung dieser Ansicht erfolgt in US 10). Die hierfür benötigte Schaltfläche findet sich in einem, nur für Admins einsehbaren, Teil des Dashboards.\\
	\hline
	Akzeptanzkriterium & Eine Liste aller angemeldeten Nutzer*innen wird nach Klicken auf die entsprechende Schaltfläche angezeigt. Dies ist nur als Administrator*in möglich. Bei Klicken auf den Namen werden die Nutzer*innendetails angezeigt (US 10).\\
	\hline
	Story Points & 3 Story Points\\
	\hline
	Entwickler & Leonhard Wiedmann\\
	\hline
	Umgesetzt in Iteration & 4\\
	\hline
	Tatsächlicher Aufwand & 1:30 h\\
	\hline
	Velocity & 0.5 h/SP\\
	\hline
	Bemerkung & Es werden keine Daten verändert, gelöscht oder erweitert. Eine Einsicht aller Nutzer*innen ist nur Administrator*innen möglich.\\
	\hline
\end{tabularx}
\vspace{20pt}
\\
	\begin{tabularx}{\textwidth}{|p{.2\textwidth}|X|}
	\hline
	ID & 10\\
	\hline
	Benutzerrolle & Admin\\
	\hline
	Name & Profil einsehen\\
	\hline
	Beschreibung & Als Admin der Seite möchte ich auf die Profile der Nutzer Einsicht haben. Ich möchte aus einer Liste von Nutzern einen Nutzer auswählen, anklicken und somit sein Profil sehen können. \\
	\hline
	Akzeptanzkriterium & Der Administrator kann jedes Profil von Nutzern einsehen, das Profil der Nutzer wird korrekt dem Admin angezeigt.\\
	\hline
	Story Points & 4 Story Points\\
	\hline
	Entwickler & Ilhan Simsiki\\
	\hline
	Umgesetzt in Iteration & \\
	\hline
	Tatsächlicher Aufwand & \\
	\hline
	Velocity & \\
	\hline
	Bemerkung & Einsicht auf die Profile fremder Nutzer hat nur ein Admin. Es werden keine Daten geändert oder eingefügt.\\
	\hline
\end{tabularx}
\vspace{20pt}
\\
	\begin{tabularx}{\textwidth}{|p{.2\textwidth}|X|}
	\hline
	ID & 11\\
	\hline
	Benutzerrolle & Nutzer*in\\
	\hline
	Name & Moderationsrechte beantragen\\
	\hline
	Beschreibung & Als Nutzer*in möchte ich auf meiner Profilseite die Möglichkeit haben, Moderationsrechte zu beantragen. Ich klicke dazu auf dem Dashboard auf die Schaltfläche ''Moderationsrechte beantragen'', woraufhin mich die Webseite auffordert, eine Begründung einzugeben. Diese wird zusammen mit einem Link auf mein Profil an die Administrator*innen geschickt.\\
	\hline
	Akzeptanzkriterium & Die Schaltfläche ist auswählbar, solange kein Antrag vorliegt. Beim Auswählen der Schaltfläche wird die Mail mit den oben genannten Informationen an die Administrator*innen verschickt. Liegt bereits innerhalb der letzten Woche ein Antrag vor, ist die Schaltfläche nicht auswählbar.\\
	\hline
	Story Points & 8 Story Points\\
	\hline
	Entwickler & Claas Völcker\\
	\hline
	Umgesetzt in Iteration & 8\\
	\hline
	Tatsächlicher Aufwand & 5.5 h\\
	\hline
	Velocity & 0.68 h/SP\\
	\hline
	Bemerkung & Es werden keine Daten geändert oder eingefügt.\\
	\hline
\end{tabularx}
\vspace{20pt}


	\chapter{Iteration}

	Diese Iteration wird dafür reserviert, den aktuellen Status der Webseite soweit zu bekommen, dass ein Alpha-Release auf dem Server der Auftraggeber*innen möglich ist. Dazu wird der Code von Back- und Frontend überarbeitet, Installskripte geschrieben und Integrationstests durchgeführt.

	\chapter{Iteration}
	\begin{tabularx}{\textwidth}{|p{.2\textwidth}|X|}
	\hline
	ID & 14 \\
	\hline
	Benutzerrolle & Nutzer*in\\
	\hline
	Name & Kurs fortführen\\
	\hline
	Beschreibung & Als Nutzer*in möchte ich einen bereits begonnenen Kurs fortführen können. Dazu gehe ich im Menü der Kursübersicht auf die Schalfläche ''Kurs fortführen'', um bei der ersten nicht beantworteten Frage fort zu fahren, oder wähle eine bereits beantwortete Frage aus, um an diesem Punkt mit der Kursbearbeitung fortzufahren. \\
	\hline
	Akzeptanzkriterium & Die Schaltfläche ''Kurs fortführen'' wird angezeigt, sobald eine Frage in einem Kurs beantwortet wurde. Es ist nicht möglich Fragen in einem Kurs zu überspringen.\\
	\hline
	Story Points & 8\\
	\hline
	Entwickler & Leonhard Wiedmann\\
	\hline
	Umgesetzt in Iteration & 7 \\
	\hline
	Tatsächlicher Aufwand & 7h \\
	\hline
	Velocity & 0,88 h/SP \\
	\hline
	Bemerkung & Es werden keine Daten in der Datenbank geändert, hinzugefügt oder gelöscht.\\
	\hline
\end{tabularx}
\vspace{20pt}
\\
	\begin{tabularx}{\textwidth}{|p{.2\textwidth}|X|}
	\hline
	ID & 15\\
	\hline
	Benutzerrolle & Nutzer (bei Registrierung)\\
	\hline
	Name & Sprachauswahl\\
	\hline
	Beschreibung & Bei der Registrierung wählt der Nutzer auch eine Sprache aus, in der alle Inhalte der Webseite angezeigt werden sollen. Danach bekommt er nur Kurse seiner gewählten Sprache angezeigt.\\
	\hline
	Akzeptanzkriterium & Die Sprachauswahl wird mit dem Nutzer gespeichert und auf der Webseite berücksichtigt. Der Nutzer erhält als  Erleichterung seiner Wahl eine Auskunft darüber, wie viele Kurse in der jeweiligen Sprache vorhanden sind.\\
	\hline
	Story Points & 6\\
	\hline
	Entwickler & Ilhan Simsiki\\
	\hline
	Umgesetzt in Iteration & 7\\ 
	\hline
	Tatsächlicher Aufwand & 8\\
	\hline
	Velocity & \\
	\hline
	Bemerkung & Aktuell werden nur Deutsch und Englisch unterstützt, aber die Möglichkeit zum Anlegen weiterer Sprachen wird im Backend berücksichtigt (Stichwort: Erweiterbarkeit)\\
	\hline
\end{tabularx}
\vspace{20pt}


	\chapter{Iteration}
	\begin{tabularx}{\textwidth}{|p{.2\textwidth}|X|}
	\hline
	ID & 12\\
	\hline
	Benutzerrolle & Admin\\
	\hline
	Name & Moderationsrechte freischalten\\
	\hline
	Beschreibung & \\
	\hline
	Akzeptanzkriterium & \\
	\hline
	Story Points & 3 Story Points\\
	\hline
	Entwickler & Tobias Huber\\
	\hline
	Umgesetzt in Iteration & \\
	\hline
	Tatsächlicher Aufwand & \\
	\hline
	Velocity & \\
	\hline
	Bemerkung & \\
	\hline
\end{tabularx}
\vspace{20pt}


	\chapter{Iteration}
	\begin{tabularx}{\textwidth}{|p{.2\textwidth}|X|}
	\hline
	ID & 19\\
	\hline
	Benutzerrolle & Nutzer*in\\
	\hline
	Name & Nutzerdetail ändern: E-Mail\\
	\hline
	Beschreibung & Als Nutzer*in möchte ich auf meinem Profil einstellen können, welche E-Mailadresse mit meinem Account verknüpft ist und dementsprechend zur Komunikation verwendet wird, indem ich im Abschnitt E-Mail das Textfeld neu ausfülle und auf "speichern" drücke.\\
	\hline
	Akzeptanzkriterium & Die Userstory wird akzeptiert, wenn der*die Nutzer*in seine E-Mailadresse ändern kann.\\
	\hline
	Story Points & 3\\
	\hline
	Entwickler & \\
	\hline
	Umgesetzt in Iteration & \\ 
	\hline
	Tatsächlicher Aufwand & \\
	\hline
	Velocity & \\
	\hline
	Bemerkung & \\
	\hline
\end{tabularx}
\vspace{20pt}
\\
	\begin{tabularx}{\textwidth}{|p{.2\textwidth}|X|}
	\hline
	ID & 20\\
	\hline
	Benutzerrolle & Nutzer*in\\
	\hline
	Name & Nutzerdetail ändern: Passwort\\
	\hline
	Beschreibung & Als Nutzer*in möchte ich auf meinem Profil unter dem Abschnitt "Passwort" mein eigenes Passwort neu festlegen können, indem ich das Textfeld neu ausfülle, mein neues Passwort noch einmal bestätige und auf "Speichern" drücke. Zur Bestätigung dieses Vorgangs muss ich noch einmal mein aktuelles Passwort eingeben.\\
	\hline
	Akzeptanzkriterium & Die Userstory wird akzeptiert, wenn der*die Nutzer*in sein*ihr Passwort ändern und sich mit diesem dann auf der Website einloggen kann.\\
	\hline
	Story Points & 4\\
	\hline
	Entwickler & Tobias Huber\\
	\hline
	Umgesetzt in Iteration & 12\\ 
	\hline
	Tatsächlicher Aufwand & 6 h\\
	\hline
	Velocity & 1.5 h/SP\\
	\hline
	Bemerkung & Die Wiederherstellung eines vergessenen Passworts wird in einer anderen Userstory behandelt.\\
	\hline
\end{tabularx}
\vspace{20pt}


	\chapter{Iteration}
	\begin{tabularx}{\textwidth}{|p{.2\textwidth}|X|}
	\hline
	ID & 4\\
	\hline
	Benutzerrolle & Moderator*in\\
	\hline
	Name & Kurs bearbeiten\\
	\hline
	Beschreibung & Als Moderator*in möchte ich meine eigenen Kurse bearbeiten können. Dafür soll in der Kursübersicht bei meinen Kursen eine Schaltfläche ''Kurs bearbeiten'' zum Anklicken sein. Nach Anklicken erscheint ein Bearbeitungsfenster, auf dem ich Titel, Zuordnung des Kurses und die einzelnen Module des Kurses bearbeiten kann. Wenn ich die Bearbeitung ohne zu speichern abbreche, wird der Kurs wieder auf seinem ursprünglichen Zustand gesetzt.\\
	\hline
	Akzeptanzkriterium & Der Editor ist für Moderator*innen des Kurses und Administrator*innen nutzbar. Er ist funktional identisch zum ''Kurs anlegen''-Editor. Der bearbeitete Kurs wird korrekt in der Datenbank abgespeichert und auf der Seite aktualisiert. Moderator*innen können nur ihre eigenen Kurs bearbeiten. Wird der Änderungsprozess abgebrochen, ändern sich die Daten in der Datenbank nicht.\\
	\hline
	Story Points & 8 \\
	\hline
	Entwickler & Leonhard Wiedmann \\
	\hline
	Umgesetzt in Iteration & 12\\
	\hline
	Tatsächlicher Aufwand & 8,25 h \\
	\hline
	Velocity & 1.03 h/SP \\
	\hline
	Bemerkung & Der bearbeitete Kurs wird in der Datenbank geändert. Dadurch werden auch abhängige Daten (Module und Frage geändert).\\
	\hline
\end{tabularx}
\vspace{20pt}
\\
	\begin{tabularx}{\textwidth}{|p{.2\textwidth}|X|}
	\hline
	ID & 17\\
	\hline
	Benutzerrolle & Nutzer*in\\
	\hline
	Name & Nutzerdetail ändern: Sprache \\
	\hline
	Beschreibung & Als Nutzer*in möchte ich auf meinem Profil einstellen können, ob ich Englisch oder Deutsch als allgemeine Sprache der Website bevorzuge, indem ich vorher einen Reiter "Einstellungen" öffne und dort im Abschnitt "Sprache" den Knopf mit der entsprechenden Sprache auswähle.\\
	\hline
	Akzeptanzkriterium & Die Userstory wird akzeptiert, wenn der*die Nutzer*in nach erfolgreicher Änderung die Website auf der korrekten Sprache angezeigt bekommt, sofern Übersetzungen vorliegen. Wird der Vorgang abgebrochen, werden keine Daten geändert.\\
	\hline
	Story Points & 3\\
	\hline
	Entwickler & Ilhan Siminski\\
	\hline
	Umgesetzt in Iteration & 10\\ 
	\hline
	Tatsächlicher Aufwand & 5 h\\
	\hline
	Velocity & 1.66 h/SP\\
	\hline
	Bemerkung & Andere Daten bleiben von den Änderungen unberührt.\\
	\hline
\end{tabularx}
\vspace{20pt}
\\
	\begin{tabularx}{\textwidth}{|p{.2\textwidth}|X|}
	\hline
	ID & 18\\
	\hline
	Benutzerrolle & Nutzer*in\\
	\hline
	Name & Nutzerdetail ändern: Name\\
	\hline
	Beschreibung & Als Nutzer*in möchte ich auf meinem Profil sowohl meinen eigenen (Vor- und Nach-) Namen, als auch meinen Username ändern können, indem ich vorher einen Reiter "Einstellungen" öffne und dort im Abschnitt "Namen" die Textfelder neu ausfülle und auf Speichern" drücke.\\
	\hline
	Akzeptanzkriterium & Die Userstory wird akzeptiert, wenn der Name in der Datenbank nach Eingeben des korrekten Passworts persistent gespeichert wird.\\
	\hline
	Story Points & 3\\
	\hline
	Entwickler & Tobias Huber\\
	\hline
	Umgesetzt in Iteration & 10\\ 
	\hline
	Tatsächlicher Aufwand & 4\\
	\hline
	Velocity & \\
	\hline
	Bemerkung & Der Username kann nicht geändert werden, da er als eindeutige Authentifikation gegenüber anderen Nutzer*innen der Plattform dient.\\
	\hline
\end{tabularx}
\vspace{20pt}


	\chapter{Iteration}
	\begin{tabularx}{\textwidth}{|p{.2\textwidth}|X|}
	\hline
	ID & 22\\
	\hline
	Benutzerrolle & Moderator\\
	\hline
	Name & Multiple Choice Bilder einpflegen\\
	\hline
	Beschreibung & Als Moderator möchte ich bei einer Multiple Choice Frage zur Frage und zum Feedback ein Bild anzeigen lassen. Dazu möchte ich als Moderator beim Fragen erstellen, die Möglichkeit haben, ein Bild für die Frage und eines für das Feedback hochzuladen. \\
	\hline
	Akzeptanzkriterium & Zum Hochladen der Bilder wird jeweils ein Popup angezeigt, in dem eine Schaltfläche angezeigt wird. Diese öffnet ein Fenster, in dem die Datei vom lokalen Rechner ausgewählt werden kann. Nach dem Hochladen wird eine Vorschau des Bildes angezeigt, auf der der gewünschte Bildausschnitt ausgewählt werden kann.
	\hline
	Story Points & 5 \\
	\hline
	Entwickler &  Leonhard Wiedmann\\
	\hline
	Umgesetzt in Iteration & 11\\
	\hline
	Tatsächlicher Aufwand & 4,5\\
	\hline
	Velocity & 1,1\\
	\hline
	Bemerkung & Die Daten werden beim Speichern des zugehörigen Kurses persistent in der Datenbank geändert.\\
	\hline
\end{tabularx}
\vspace{20pt}
\\
	\begin{tabularx}{\textwidth}{|p{.2\textwidth}|X|}
	\hline
	ID & 16\\
	\hline
	Benutzerrolle & Nutzer\\
	\hline
	Name & Kursfortschritt anzeigen (Übersicht)\\
	\hline
	Beschreibung & Als Nutzer möchte ich meinen Fortschritt in einem Kurs in der Kursübersicht einsehen können.\\
	\hline
	Akzeptanzkriterium & Die Schaltfläche zur Kurswahl wird prozentual zum Fortschritt eines Nutzers im jeweiligen Kurs eingefärbt. In der Kursübersicht wird die Anzahl der beantworteten Fragen und die Gesmatzahl an Fragen im Kurs angezeigt.\\
	\hline
	Story Points & 5\\
	\hline
	Entwickler & Leonhard Wiedmann\\
	\hline
	Umgesetzt in Iteration & \\
	\hline
	Tatsächlicher Aufwand & \\
	\hline
	Velocity & \\
	\hline
	Bemerkung & Ist noch nicht begonnen, nur formuliert.\\
	\hline
\end{tabularx}
\vspace{20pt}


	\chapter{Iteration}
	\begin{tabularx}{\textwidth}{|p{.2\textwidth}|X|}
	\hline
	ID & 23\\
	\hline
	Benutzerrolle & Moderator*in\\
	\hline
	Name & Einklappen von Modul Komponenten ermöglichen\\
	\hline
	Beschreibung & In der Kursbearbeiten-Übersicht soll jede Komponente Module ein und ausklappbar sein.
		Dazu soll neben der Überschrift der jeweiligen Komponente eine Schaltfläche vorhanden sein.
		Ein Klick auf diese ermöglicht, abhängig vom aktuellen Zustand, das Ein- und Ausklappen. \\
	\hline
	Akzeptanzkriterium & Die Schaltfläche ist für jede Komponete vorhanden und durch das Klicken wird nur die betreffende Komponente animiert.\\
	\hline
	Story Points & 8 \\
	\hline
	Entwickler & Ilhan Siminski\\
	\hline
	Umgesetzt in Iteration & 18\\
	\hline
	Tatsächlicher Aufwand & 8 h\\
	\hline
	Velocity & 1 h/SP\\
	\hline
	Bemerkung & Es werden keine Daten geändert, neu hinzugefügt oder gelöscht. Die Fragen übersichtlicher zu machen wurde in US 58 ausgelagert.\\
	\hline
\end{tabularx}
\vspace{20pt}
\\
	\begin{tabularx}{\textwidth}{|p{.2\textwidth}|X|}
	\hline
	ID & 24 \\
	\hline
	Benutzerrolle & Moderator \\
	\hline
	Name & Info Fragetyp einpflegen\\
	\hline
	Beschreibung & Als Moderator möchte ich einen Informationstext anstelle einer Frage einfügen können.
		Dieser soll im Editor wie eine normale Frage anlegbar und bearbeitbar sein.
		Anstelle von möglichen Antworten kann ein längerer Text eingefügt werden.\\
	\hline
	Akzeptanzkriterium & Der Info-typ kann im Editor wie eine Frage hinzugefügt und bearbeitet werden und wird nach speichern persistent in der Datenbank angelegt. \\
	\hline
	Story Points & 5 \\
	\hline
	Entwickler & Claas Völcker\\
	\hline
	Umgesetzt in Iteration & \\ 
	\hline
	Tatsächlicher Aufwand & \\
	\hline
	Velocity & \\
	\hline
	Bemerkung & \\
	\hline
\end{tabularx}
\vspace{20pt}
\\
	\begin{tabularx}{\textwidth}{|p{.2\textwidth}|X|}
	\hline
	ID & 25\\
	\hline
	Benutzerrolle & Nutzer*in\\
	\hline
	Name & Info Fragetyp anzeigen\\
	\hline
	Beschreibung & Als Nutzer*in möchte ich beim Bearbeiten des Kurses Fragen vom Typ Information angezeigt bekommen.\\
	\hline
	Akzeptanzkriterium & Dem*der Nutzer*in werden Fragen vom Typ Information angezeigt und er kann sie bearbeiten, indem er auf die Schlatfläche "Fortfahren" klickt. Die zu jeder Frage gehörenden Informationen werden im linken Drittel des Fensters und der Text der Frage auf den rechten zwei Dritteln angezeigt.\\
	\hline
	Story Points & 5 \\
	\hline
	Entwickler & Claas Völcker \\
	\hline
	Umgesetzt in Iteration & 14\\
	\hline
	Tatsächlicher Aufwand & 5,5 h\\
	\hline
	Velocity & 1.1 h/SP\\
	\hline
	Bemerkung & Durch Auswählen der Schlatfläche ''Fortfahren'' wird die Frage in der Statistik des*der Nutzer*in gespeichert.\\
	\hline
\end{tabularx}
\vspace{20pt}
\\
	\begin{tabularx}{\textwidth}{|p{.2\textwidth}|X|}
	\hline
	ID & 26\\
	\hline
	Benutzerrolle & Nutzer*in\\
	\hline
	Name & Informationsseite\\
	\hline
	Beschreibung & Als Nutzer*in möchte Ich mir Informationen, über die Webseite und das zugrunde liegende iGem Projekt einholen, indem Ich im entsprechenden Reiter auf den Button "Introduction" drücke.\\
	\hline
	Akzeptanzkriterium & Die Seite wird korrekt dargestellt.\\
	\hline
	Story Points & 2\\
	\hlinet
	Entwickler & \\
	\hline
	Umgesetzt in Iteration & \\ 
	\hline
	Tatsächlicher Aufwand & \\
	\hline
	Velocity & \\
	\hline
	Bemerkung & \\
	\hline
\end{tabularx}
\vspace{20pt}

	
	\chapter{Iteration}


\chapter*{Noch geplant, nicht begonnen}

	\begin{tabularx}{\textwidth}{|p{.2\textwidth}|X|}
	\hline
	ID & 16\\
	\hline
	Benutzerrolle & Nutzer\\
	\hline
	Name & Kursfortschritt anzeigen (Übersicht)\\
	\hline
	Beschreibung & Als Nutzer möchte ich meinen Fortschritt in einem Kurs in der Kursübersicht einsehen können.\\
	\hline
	Akzeptanzkriterium & Die Schaltfläche zur Kurswahl wird prozentual zum Fortschritt eines Nutzers im jeweiligen Kurs eingefärbt. In der Kursübersicht wird die Anzahl der beantworteten Fragen und die Gesmatzahl an Fragen im Kurs angezeigt.\\
	\hline
	Story Points & 5\\
	\hline
	Entwickler & Leonhard Wiedmann\\
	\hline
	Umgesetzt in Iteration & \\
	\hline
	Tatsächlicher Aufwand & \\
	\hline
	Velocity & \\
	\hline
	Bemerkung & Ist noch nicht begonnen, nur formuliert.\\
	\hline
\end{tabularx}
\vspace{20pt}
\\


\end{document}
