\documentclass[colorback, accentcolor=tud1c, paper=a4]{tudexercise}
\usepackage[ngerman]{babel}
\usepackage[utf8]{inputenc}
\usepackage{multicol}

%opening
\title{Protokol Auftraggebertreffen 12.05.2017}
\subtitle{Claas A. Völcker}
\subsubtitle{BP Clonecademy}
\author{Claas A. Völcker}


\begin{document}
	
	\maketitle
\begin{multicols}{2}
\subsection*{Technische Fragen}
	\begin{itemize}
	\item Multi device Nutzung
		\begin{itemize}
		\item nur Desktop Nutzung
		\end{itemize} 	
	\item Multilingualität
		\begin{itemize}
		\item Fragen werden Kursen in einer Sprache zugeordnet, um mehrere Sprachen zu ermöglichen, muss der Kurs zweimal angelegt werden.
		\item Das Interface wird mit Option auf Mehrsprachigkeit (Deutsch/Englisch) angelegt
		\end{itemize}
	\item Die Webseite wird mit Bootstrap erstellt
	\item Die BP-Gruppe erhält keinen Admin Zugang zum Server des iGEM Teams
	\end{itemize}
\subsection*{Inhaltliche Fragen}
	\begin{itemize}
	\item Welche Kategorisierungen von Fragen gibt es?
	\begin{itemize}
		\item Es gibt Kurse, die aus einer unterschiedlichen Menge an Fragen bestehen.
		\item Die Kurse sind nach Schwierigkeit kategorisiert:Anfänger*innen Schule/Studium - Expert*innen
		\item Jeder Kurs erhält optional Tags mit den Themen des Kurses (z.B.: Klonierungsverfahren)
	\end{itemize}
	\item Kurse sind nach dem Anlegen nicht sofort sichtbar, sondern müssen erst freigeschaltet werden. Das erlaubt es, den Kurs weiter zu anzupassen, bevor Lernenden diesen bearbeiten.
	\item Alle Fragen bestehen aus mindestens einem Namen und gehören zu einem Kurs.
	\item Persistent gespeichert werden soll
		\begin{itemize}
		\item Der Fortschritt eines*r Nutzer*in im Kurs
		\item Die vollständig bearbeiteten Kurse
		\end{itemize}
	\item  Wie soll die Fragenseite aufgebaut sein?
		\begin{itemize}
		\item Analog zu Codecademy: Links die Aufgabenstellung \& weitere Informationen,  Rechts das Bearbeitungsfenster
    	\item Beim Feedbackfenster wird Fragestellung \& Antwortmöglichkeiten mit angezeigt
    	\item Ein Knopf zum Anzeigen eines Hinweises sollte eingebaut werden.
		\end{itemize}
	\item Aufbau der Landing Page / des Dashboards
		\begin{itemize}
		\item Wie bei Codecademy mit aktiven Kursen, Navigationsschaltflächen und weiteren Kursen
		\end{itemize}
	\item Welche Nutzerrollen gibt es?
		\begin{itemize}
		\item Identifiziert sind Admin (Zugriff auf alles), Moderator (stellt Kurse ein und gibt diese frei)
		\end{itemize}
	\item Kein dynamisches Feedback, nur statische: Es war falsch, so ist der richtige Ansatz
	\item Kurse sollen nach Schwierigkeit und Kategorien filterbar sein
	\item Ob Nutzer*innen am Ende eines Kurses eine Möglichkeit zum Feedback gegeben wird, wird noch diskutiert.
	\end{itemize}
		
\subsection*{Weiteres Vorgehen}
\begin{itemize}
\item Ein PDF mit allen (vollständigen) Userstories (ToDo, Diese Iteration) immer bis Montag
  im Seafile hochladen
\end{itemize}
\end{multicols}
	
\end{document}