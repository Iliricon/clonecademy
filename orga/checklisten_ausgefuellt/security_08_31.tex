\begin{itemize}

In dieser Iteration bearbeitete Userstories:
\begin{itemize}
\item 30 - Moderatorenrechte entziehen
\item 32 - Adminrechte entziehen
\item 36 - Footer auf der Hauptseite
\item 37 - Footer auf der Frageseite
\item 40 - InfoText mit Youtube Video anzeigen
\item 41 - InfoText mit Youtube Video einpflegen
\end{itemize}

\item HTML Input: Wird jeglicher Input korrekt nach Cross-Site-Scripts durchsucht, bevor er als sicher markiert wird?

\emph{Für die US 40 und 41}
Ja, sämtlicher Input wird korrekt validiert.

\emph{Für die US 30, 32, 36 und 37}
Kein Input vorhanden.


\item Sind die Django-Features zum Schutz vor Cross Site Request Forgery (CSRF) aktiviert?
ges
Ja, genauer gesagt, wurden die Features nicht dektiviert.


\item Ist jede Schnittstelle mit passender Authentifizierung und Autorisierung versehen, sowie gegebenenfalls auch mit einer Anfragenbegrenzung (Throttling)?

\emph{Für die Userstories 36, 37, 40 und 41}
Nein, doch das war Absicht in dieser Iteration, da die Aufgabe komplett in den Userstories 30 bis 32 aufgeht.
\emph{Für die Userstories 30 und 32}
Ja, jedoch fehlt noch die Userstory 31, damit alle Funktionalitäten korrekt umgesetzt sind.


\item Werden alle Nutzereingaben nach Fehlern gefiltert?

\emph{Für die US 40 und 41}
Ja, sämtlicher Input wird korrekt validiert.

\emph{Für die US 30, 32, 36 und 37}
Kein Input vorhanden.

\item Sind alle Security Probleme, die bei der softwarebasierten Analyse aufgetreten sind, behoben, oder aus dem Kontext heraus als unkritisch betrachhtet worden?

Ja, einige ng-Fehler wurden behoben. Vier aufgetretene ZAP-Alerts sind Browserseitig zu beheben und erforedern keine Änderung des Quellcodes.


\end{itemize}

