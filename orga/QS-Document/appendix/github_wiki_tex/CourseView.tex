 \chapter*{Course View}

\section*{Introduction}\label{introduction}

The CourseView class is used to get information for a specific course.
Courses are referenced via the general url ``clonecademy.com/courses/''.

\section*{Method Details}\label{method-details}

\subsection*{\texorpdfstring{`get'}{get}}\label{get}

Header: user id (for permissions), course id (from url)

Response:

\begin{verbatim}
{
"name": string "Course name";
"difficulty": number 0/1/2/3;
"category": string "iGEM/General";
"language": string "de/en";
"modules": [{
              "name": string "Module name";
              "learning_text": string "Sample learning text";
              "questions": [{
                              "type": string "MultipleChoiceQuestion";
                              "solved": boolean True/False;
                            }];
            }];
}
\end{verbatim}

\subsection*{\texorpdfstring{`post'}{post}}\label{post}

Header: user id (for permissions)

Request:

\begin{verbatim}
{
"name": string "Course name",
"difficulty": number 0/1/2/3,
"modules": [{
              "name": string "Module name",
              "learning_text": string "Sample learning text";
              "questions": [{
                              "type": string "MultipleChoiceQuestion";
                              "question_title": string "(currently blank)";
                              "question_body": string "The text for the question";
                              "feedback": string "A specific feedback text (leave empty for standard)"
                            }]
            }]
"quiz": [{
        "question" : string,
        "image": string,
        "answers": [{
                    "text": string,
                    "img": string,
                    "correct": boolean,
                    }]
         }]
}
\end{verbatim}

Response: 201 (saved), 500 + serializer error (error)