\documentclass[accentcolor=tud0b,12pt,paper=a4]{tudreport}

\usepackage[utf8]{inputenc}
\usepackage{ngerman}
\usepackage{parcolumns}
\usepackage{hyperref}
\usepackage{multicol}

\setlength{\parindent}{0pt}
\setlength{\parskip}{1em}

\newcommand{\titlerow}[2]{
	\begin{parcolumns}[colwidths={1=.15\linewidth}]{2}
		\colchunk[1]{#1:}
		\colchunk[2]{#2}
	\end{parcolumns}
	\vspace{0.2cm}
}

\title{CloneCademy}
\subtitle{Qualitätssicherungsdokument}
\subsubtitle{%
	\titlerow{Gruppe}{%
		Ilhan Simsiki
		<\href{mailto:ilhan.simsiki@stud.tu-darmstadt.de}{ilhan.simsiki@stud.tu-darmstadt.de}>\\
		Leonhard Wiedmann
		<\href{mailto:leonhard.wiedmann@stud.tu-darmstadt.de}{leonhard.wiedmann@stud.tu-darmstadt.de}>\\
		Tobias Huber
		<\href{mailto:tobias.huber@stud.tu-darmstadt.de}{tobias.huber@stud.tu-darmstadt.de}>\\
		Claas Völcker
		<\href{mailto:c.voelcker@stud.tu-darmstadt.de}{c.voelcker@stud.tu-darmstadt.de}>}
	\titlerow{Teamleiter}{Alexander Nagl
		<\href{mailto:alexander.nagl@t-online.de}{alexander.nagl@t-online.de}>}
	\titlerow{Auftraggeber}{%
		iGEM Team TU Darmstadt\\
		Thea Lotz <\href{mailto:lotz@bio.tu-darmstadt.de}{lotz@bio.tu-darmstadt.de}>\\
		Fachbereich Biologie}
	\titlerow{Abgabedatum}{xx.xx.xxxx}}
\institution{Bachelor-Praktikum SoSe 2017\\Fachbereich Informatik}

\begin{document}

\maketitle
\tableofcontents

\chapter*{Anmerkungen und Hinweise}
Die Verfasser dieses Dokumentes verwenden als Maßnahme für die Gleichstellung aller Geschlechter die sogenannte ''gendergerechte Sprache''. Wir weichen nur von dieser Regelung ab, wenn entweder alle Mitglieder einer Gruppe sich ein und demselben Geschlecht zugehörig fühlen oder die Vorgaben der Veranstalter*innen keine Alternative zulassen (z.B. beim Deckblatt).

\chapter{Einleitung}

\emph{CloneCademy} ist ein Projekt für das iGEM-Team der TU Darmstadt. Die \emph{international Genetically Engineered Machine competition} (iGEM) ist ein internationaler Wettbewerb für Studierende auf dem Gebiet der Synthetischen Biologie.
Dieser wird seit 2003 von der iGEM-Foundation veranstaltet.

Im Rahmen des Wettbewerbs wird eine Online-Lernplattform für das iGEM-Team der TU Darmstadt erstellt. Das Ziel dieser Plattform ist es, durch interaktive Unterrichtseinheiten Prinzipien der Molekularbiologie sowie der synthetischen Biologie zu erlernen, und sowohl eigene Lernfortschritte als auch die anderer Teams, begutachten zu können. Darüber hinaus soll es auch anderen Interessierten (z.B. andere iGEM-Teams, Lehrende an Universitäten und Schulen, etc.) möglich sein, eigene Inhalte einzupflegen und zur Verfügung zu stellen.

Das Ziel des Projektes ist eine voll funktionsfähige Webanwendung für Rechner. Eine mobile Nutzung der Seite ist nicht Teil des Bachelor Praktikums, eine spätere Implementierung einer plattformübergreifenden Lösung (z.B. einer App) ist jedoch durch Nutzung der REST-Schnittstelle möglich.

Der minimale geplante Funktionsumfang umfasst die Möglichkeit einen Nutzungsaccount auf der Seite anzulegen und eingestellte Kurse zu bearbeiten. Außerdem wird es dedizierte Moderator*innen geben, die Inhalte auf der Webseite hochladen können.

Zusätzlich sind mehrere Erweiterungen geplant. Allem voran soll es möglich sein, unterschiedliche Fragetypen zu nutzen. Als minimales Ziel wird hierfür die Möglichkeit gegeben, Multiple Choice Fragen zu stellen. Weitere Arten von Fragen werden im Laufe des Projektes mit den Auftraggeber*innen besprochen und beschlossen.

Da auch die Interaktion verschiedener Nutzer*innen und der Lernfortschritt erklärte Ziele der Plattform sind, werden zusätzliche Erweiterungen indiesen Gebeten geplant. Dazu zählen ausführliche Statistiken zu Lernfortschritt und die Möglichkeit, diesen mit anderen Nutzer*innen der Plattform zu teilen und zu vergleichen.

Das erste Release auf einem Server der Auftraggeber*innen wird Ende Juli/Anfang August stattfinden. Bis dahin soll das oben skizierte Minimale Produkt lauffähig und vollständig sein. Weitere Releases werden dann abhängig von der Geschwindigkeit des Teams geplant.


\chapter{Qualitätsziele}
\section{Datensicherheit (Security)}

Im Rahmen des Projekts CloneCademy wird eine Webanwendung entwickelt, auf die über das Internet zugegriffen werden kann. Daher ist die Sicherung gegen unbefugten Zugriff und Änderung der Daten der Webseite in diesem Projekt das wichtigste Qualitätsziel. Da CloneCademy sowohl persönliche Daten der Nutzer*innen als auch Metadaten über die Nutzung der Plattform und die Inhalte der einzelnen Lerneinheiten in einer Datenbank speichert, ist es sehr wichtig, dass Internetnutzer*innen keine Daten verändern oder einsehen können, solange sie dazu nicht die benötigten Rechte besitzen.

Die größte Bedrohung geht von bekannten Webangriffen und Misskonfigurationen im Backend einer Webseite aus. Während dieses Projekts wird darauf geachtet, dass wir die Plattform gegen die wichtigsten Sicherheitslücken in einer Webanwendung sichern. Da eine umfassende Sicherung gegen alle möglichen Angriffsvektoren und Schwachstellen den Projektumfang weit übersteigen würde, befassen wir uns im Rahmen dieses Projektes vor allem mit der Sicherheit der Nutzeroberfläche und der Schnittstelle.

Die betrachtenten Angriffsvektoren\footnote{Paraphrasiert nach der verbreiteten Liste des Open Web Application Security Project (OWASP)\\Link zum Download:  \href{https://github.com/OWASP/Top10/raw/master/2017/OWASP\%20Top\%2010\%20-\%202017\%20RC1-English.pdf}{https://github.com/OWASP/Top10/raw/master/2017/OWASP\%20Top\%2010\%20-\%202017\%20RC1-English.pdf}} sind:
\begin{itemize}
\item Möglichkeiten zur Ausführung fremden Codes (Injection \& Cross-Site Scripting)
\item Fehlerhafte Zugriffskontrolle
\item Nutzung von Komponenten mit bekannten Schwachstellen
\end{itemize}

Um die Sicherheit der Anwendung zu gewährleisten, folgen wir einem mehrschrittigen Plan.

Ein Entwickler des Teams wurde zum Sicherheitsbeauftragten ernannt, der die korrekte Durchführung aller genannten Maßnahmen sicher stellt. Seine Aufgabe ist es, auf die Einhaltung aller Programmierrichtlinien zu achten und Tests zur Validierung der Schutzziele durchzuführen. Die genaue Vorgehensweise wird im Folgenden detailliert ausgeführt.

Während der ersten Gespräche mit den Auftraggeber*innen wurden Nutzerrollen definiert (Admin, Moderator*in und Nutzer*in)\footnote{Detaillierte Beschreibung der Rollen finden sich im Anhang}. In jeder User Story wird festgehalten, ob und welche Zugriffsbeschränkungen einzelne Module der Webseite oder Daten besitzen. Um die Einhaltung der Nutzer*innenrollen zu ermöglichen, werden die zur Verfügung gestellten Authentifizierungswerkzeuge der genutzten Frameworks (Django, Django REST, Angular2) verwendet.

Während der Implementierung halten alle Entwickler die Best Practices im Web Development ein. Als Referenz dafür werden die Handreichungen des Open Web Application Security Project (OWASP)\footnote{\href{https://www.owasp.org/images/0/08/OWASP_SCP_Quick_Reference_Guide_v2.pdf}{https://www.owasp.org/images/0/08/OWASP\_SCP\_Quick\_Reference\_Guide\_v2.pdf}} verwendet. In den wöchentlichen internen Code-Reviews wird die Qualität des Codes hinsichtlich dieser Richtlinien überprüft.

Um die Qualität des Codes im Python-basierten Backend automatisiert zu testen, wird das verbreitete Werkzeug \emph{bandit}\footnote{\href{https://wiki.openstack.org/wiki/Security/Projects/Bandit}{https://wiki.openstack.org/wiki/Security/Projects/Bandit}} verwendet. Als statisches Analyse-Werkzeug für das Frontend wird \emph{ng2lint}\footnote{\href{https://www.npmjs.com/package/ng2lint}{https://www.npmjs.com/package/ng2lint}} verwendet. Beide Programme unterstützen die Code-Review, indem die gelieferten Meldungen als Grundlage für eine detaillierte Analyse des Codes genutzt werden. Zusätzlich wird eine Sicherheits-Checkliste\footnote{siehe Anhang} geführt, die die Best Practices zur sicheren Webentwicklung und das Überprüfen aller Reports der verwendeten Werkzeuge enthält. Um eine dauerhaft hohe Qualität des Codes zu gewährleisten, werden die Analyse-Tools automatisiert bei einem Push auf dem Development Branch des Git Repositories verwendet. Alle gefundenen Schwachstellen werden vor dem finalen Commit auf dem stable Branch zur Abnahme der User Story behoben. Wenn dies nicht möglich ist, wird eine Begründung verfasst und eine realistische Einschätzung über die Risiken der gefundenen Fehler an die Auftraggeber*innen gemeldet. Diese entscheiden, ob die Risiken tragbar sind und die User Story trotz der Sicherheitslücke als erfolgreich gemeldet wird.

Um die tatsächliche Sicherheit des Endproduktes gegen externe Angriffe zu zeigen, wird ein automatisiertes Penetrations-Werkzeug verwendet. Das OWASP empfiehlt hierfür das \emph{Zed Attack Proxy Project}, welches eine Reihe bekannter Angriffe auf die Plattform ausführt und die gefundenen Schwachstellen meldet. Da ein gesamter Test der Anwendung sehr zeitaufwändig ist, führt der Sicherheitsbeauftragte diese Tests monatlich durch und meldet die Ergebnisse an die jeweiligen Entwickler des betroffenen Moduls, damit diese innerhalb des nächsten Monats behoben werden können. Ist eine Behebung nicht möglich, werden die sicherheitsrelevanten Gefahren dieser Lücke zusammen mit möglichen weiteren Schritten zum Schutz der Anwendung im Betrieb an die Auftraggeber*innen gemeldet.

Da ein Fokus auf Sicherheit zwar wichtig ist, aber Korrekturen erst vorgenommen werden können, sobald das Kernprodukt nicht mehr verändert wird, werden alle vorgestellten Maßnahmen ab Juli durchgeführt.

\section{Bedienbarkeit (Entwurf)}
Clonecademy ist eine Lernplattform und steht als solche einer großen Vielfalt an Benutzer*innen zur Verfügung. Daher muss die Plattform für alle Benutzer*innen einfach und intuitiv bedienbar sein, ohne das eine Einleitung notwendig ist. Um dies zu gewährleisten, wird das Design aller Bereiche der Webseite nach diesem Kriterium evaluiert. Die grundlegende Bedienung der Webseite wird ohne Einführung erlernbar sein und komplizierte Elemente werden mit einer ausreichenden Erklärung versehen.

Auch für dieses Ziel wurde ein Mitglied des Teams benannt, um Nutzungsstudien durchzuführen und auf die Einhaltung aller Richtlinien zu achten.

Um das oben formulierte Ziel zu erreichen, verwenden wir als Grundlage für das Design der Seite das Material Design\footnote{href{https://material.angular.io/}{https://material.angular.io/}} für Angular2. Dies sorgt dafür, dass die Webseite einen einheitlichen und übersichtlichen Aufbau hat. Um Knöpfe und andere Schaltflächen übersichtlicher zu machen, werden diese mit Icons\footnote{\href{https://material.io/icons/}{https://material.io/icons/}} versehen. Bei der internen Code Review prüft der zuständige Entwickler bei allen veränderten oder neuen Bereichen der Webseite, ob sie noch den Richtlinien eines einheitlichen Designs entsprechen\footnote{Checkliste siehe Anhang}.


Da die Bedienbarkeit der Oberfläche nicht objektiv gemessen werden kann, werden zuzüglich zu den Designrichtlinien Nutzungsstudien durchgeführt. In den Studien werden die Proband*innen ohne vorherige Erklärung mit der Platform konfrontiert und aufgefordert, verschiedene Aufgaben zu erfüllen. Durch Aufzeichnen des Bildschirms während der Studie wird das Verhalten der Nutzer*innen erfasst um es später zu reflektieren und mit einem Fragebogen\footnote{siehe Anhang} nach der Studie wird die persönliche Meinung der Proband*innen zusätzlich erfasst.

Um zum Schluss ein von einer Vielzahl unterschiedlicher Nutzer*innen gut bedienbares Produkt zu haben, werden die Nutzungsstudien in mehreren Iterationen und mit unterschiedlichen Gruppen durchgeführt, um Verbessungsvorschläge aus den vorherigen Studien direkt wieder zu testen. 

Nach jeder Nutzerstudie wird eine Checkliste erstellt mit den Problemen und Verbesserungsvorschläge der Probanden, welche bis zum nächsten Treffen mit den Auftraggeber*innen umgesetzt werden muss. Diese können dann die Designänderungen überprüfen und erst durch die Abnahme der Designänderungen ist die Iteration der Studie abgeschlossen.

Die Nutzerstudien erfolgen mindestens eine Woche vor den Releases und wenn die Auftraggeber*innen eine Zwischenevaluation des Designs wünschen.

\section{Veränderbarkeit (Entwurf)}
Für die Webanwendung CloneCademy ist es wichtig, dass die Anwendung im Nachhinein noch veränderbar ist. Es muss möglich sein, neue Inhalte in die Datenbank einzupflegen und auch den Quellcode der Webseite selbst ohne große Mühe erweitern zu können, da eine Lernplattform immer an die neusten Entwicklungen im Bereich eLearning angepasst werden muss.

Um diese Erweiterbarkeit zu gewährleisten, wird im Projekt auf folgende Punkte geachtet:
\begin{itemize}
	\item Quellcode
	\item Kommentare
	\item externe Dokumentation
\end{itemize}

Es wird ein Verantwortlicher für das Qualitätsziel benannt, der Ansprechpartner für alle Rücksprachen oder Fragen ist und dafür sorgt, dass die QUlaität der oben aufgeführten Bereiche wie unten beschrieben sichergestellt wird.

\textbf{Qualität des Quellcodes}
Grundlegend werden während der Entwicklung der Software die StyleGuides der verwendeten Programmiersprachen und Frameworks umgesetzt. Diese sind der \emph{Angular Style Guide}\footnote{https://angular.io/guide/styleguide} für das Frontend und \emph{PEP 8}\footnote{https://www.python.org/dev/peps/pep-0008/} für das Backend. Um die Einhaltung dieser beiden Style Guides zu überprüfen, werden bei jedem Commit statische Analysetools verwendet, welche überprüfen, ob alle Richtlinien umgesetzt wurden.

Diese sind \emph{pep8}\footnote{https://pypi.python.org/pypi/pep8}, welches die Einhaltung der generellen Python Richtlinien überprüft, \emph{django-lint}\footnote{https://pypi.python.org/pypi/django-lint}, welches Framework spezifische Fehler markiert (z.B. Verwendung von veralteten Methoden) und \emph{ng2lint}\footnote{https://www.npmjs.com/package/ng2lint}, welches den Frontend Code überprüft. Alle durch diese Tools gefundenen Fehler werden bis zum nächsten internen Gruppentreffen der Entwickler behoben, bevor der Code zur Abnahme durch die Auftraggeber*innen freigegeben wird.

Bei der internen Code Review vor der Abnahme einer User Story überprüft der für die Erweiterbarkeit zuständige Entwickler, ob alle von den Tools gemeldeten Fehler korrigiert wurden. Sollte dies nicht der Fall sein, müssen alle gefundenen Fehler korrigiert werden, bevor die Userstory dem Auftraggeber zur Abnahme vorgestellt wird.


\textbf{Kommentare}
Jede Methode und jede Klasse im Code erhält einen Kommentar. Zusätzlich werden unklare Stellen im Code (z.B. komplexe Steuermechanismen oder Abfragen) mit eigenen Erklärungen versehen. Alle Kommentare werden in Englisch verfasst.

Bei der Code Review überprüft der Beauftragte für das QS-Ziel, ob alle Klassen und Methoden kommentiert sind und ob die Kommentare das Verständnis des Codes unterstützen\footnote{Checkliste siehe Anhang}. Zusätzlich markieren alle an der Code Review beteiligten Entwickler die Stellen im Code, die sie nicht für unmittelbar verständlich halten. Diese werden auch mit Kommentaren versehen. Sollte es nicht möglich sein, den gesamten Code ausführlich zu dokumentieren, wird die Userstory den Auftraggeber*innen in dieser Iteration nicht zur Abnahme präsentiert.


\textbf{Wiki}
Das Projekt verwendet zu klaren Trennung zwischen Back- und Frontend eine sogenannte REST-API. Diese stellt einzelne Schnittstellen da, über die die Daten des Backends abgerufen werden können. Diese werden deshalb gesondert dokumentiert, da eine korrekte Implementierung und Nutzung essentiel für das Projkt ist. 

Dazu wird ein Wiki regelmäßig aktualisiert, in dem alle Schnittstellen zwischen Backend und Frontend definiert sind. Ist für eine Userstory eine neue Schnittstelle nötig, wird diese von allen Entwicklern zusammen entworfen und dann im Wiki dokumentiert\footnote{Rahmen im Anhang}. Die Beteiligung aller ist notwendig, da einen Schnittstelle nicht nur für eine spezifische Aufgabe im Rahmen einer Userstory dienen sollte, sondern allgemeine Funktionalitäten bietet.

Die Dokumentation im Wiki wird spätestens vor den Releases vervollständigt. Dazu überprüft der Beauftragte spätestens zwei Wochen vor Release, ob alle Schnittstellen aufgeführt sind, und ob die Dokumentation den vorgegebenen Rahmen erfüllt. Beim nächsten internen Treffen wird das Wiki auf den aktuellen Stand gebracht und alle Dokumentatoin noch einmal überprüft. Bevor die Dokumentation nicht vollständig ist, wird die Software nicht zum Release freigegeben.


\textbf{Testabdeckung}
Bei Codeerweiterung muss bereits bestehender Code weiterhin funktionieren. Um dies automatisiert zu überprüfen, muss der bereits implementierte Code getestet werden können. Das Team stellt eine umfangreiche Testabdeckung des Codes sicher. Umfangreich heißt in diesem Fall, dass eine vollständige Function Coverage und eine Statement Coverage von mindestens 80\% erreicht wird. Eine Überprüfung der Abdeckung erfolgt im Backend durch das mitgelieferte Testframework des Django-Projekts. Im Frontend werden Tests mit Hilfe des Frameworks Selenium\footnote{http://www.seleniumhq.org/} durchgeführt.

Die Function Coverage wird vor jeder internen Code Review sichergestellt. Die Tests der API Schnittstellen werden bei der Definition der selben geschrieben, damit sie den jeweiligen Entwicklern als Hilfestellung dienen können. Da eine umfassende Statement Coverage während des Projektes nicht zu jeder Zeit implementiert werden kann, wird diese zu jedem Release gefordert. Damit wird gezeigt, dass der Code stabil ist und die Tests können als Basis für die Implementierung der kommenden Userstories verwendet werden.

Eine Userstory wird erst zur Abnahme präsentiert, wenn mindestens für alle durch diese Story verwendeten und neu implementierten Funktionen Tests vorliegen. Ein Release wird erst freigegeben, wenn eine Code Coverage von mindestens 80\% erreicht ist. Die Überprüfung dieser Regeln erfolgt durch den für die Bedienbarkeit zuständigen Entwickler.


\appendix
\chapter{Checklisten}
\renewcommand{\labelitemi}{\scriptsize$\square$}
Im folgenden sind die verwendeten Checklisten aufgeführt. Für die Ziele \emph{Datensicherheit} und \emph{Erweiterbarkeit} haben wir einen Checkliste für die interne Code Review ausgearbeitet. Für die Bedienbarkeit wird keine Checkliste bereitgehalten, sondern die Checklisten ergeben sich aus den Rückmeldungen aus den Nutzungsstudien.

\section{Checkliste Datensicherheit}


\section{Checkliste Erweiterbarkeit}
\begin{multicols}{2}
	\subsection{Code Qualität}

	\subsection{Kommentare}

	\subsection{Tests}

\end{multicols}

\chapter{Fragebögen}

\end{document}
