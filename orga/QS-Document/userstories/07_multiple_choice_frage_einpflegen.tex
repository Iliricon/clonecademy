\begin{tabularx}{\textwidth}{|p{.2\textwidth}|X|}
	\hline
	ID & 7\\
	\hline
	Benutzerrolle & Moderator*in\\
	\hline
	Name & Multiple Choice Fragen einpflegen\\
	\hline
	Beschreibung & Als Moderator*in kann ich einem Kurs neue Fragen hinzufügen. Dazu wähle ich auf der Kursübersicht die Schaltfläche ''Neue Frage anlegen'' aus. Ich kann daraufhin in einem Interface den Name der Frage, den Text der Frage und mögliche Antwortmöglichkeiten eingeben. In einem weiteren Fenster kann ich einen Text eingeben, der den Nutzenden bei erfolgreicher Bearbeitung angezeigt wird und einen Feedbacktext bei falscher Beantwortung.\\
	\hline
	Akzeptanzkriterium & Eine neu angelegte Frage wird persistent unter dem Kurs gespeichert und ist nach Freigabe für Nutzende bearbeitbar.\\
	\hline
	Story Points & 8 Story Points\\
	\hline
	Entwickler & Leonhard Wiedmann\\
	\hline
	Umgesetzt in Iteration & 4\\
	\hline
	Tatsächlicher Aufwand & 15 h\\
	\hline
	Velocity & 1.875 h/SP\\
	\hline
	Bemerkung & Moderator*innen können nur Fragen in ihren eigenen Kursen anlegen. Es werden neue Daten in die Datenbank geschrieben, aber keine existierenden verändert oder gelöscht.\\
	\hline
\end{tabularx}
\vspace{20pt}
