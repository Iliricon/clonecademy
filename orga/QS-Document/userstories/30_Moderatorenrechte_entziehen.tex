\begin{tabularx}{\textwidth}{|p{.2\textwidth}|X|}
	\hline
	ID & 30 \\
	\hline
	Benutzerrolle & Admin \\
	\hline
	Name & Moderatorenrechte entziehen\\
	\hline
	Beschreibung & Als Admin möchte ich einem Moderator seine Moderatorenrechte wieder entziehen können. Der Moderator soll dann zu einem normalen Nutzer werden. \\
	\hline
	Akzeptanzkriterium & Im Profil des*der Moderator*in ist für Administartor*innen eine Schaltfläche vorhanden, die es ermöglicht, die Moderationsrechte zu entziehen. Nach Klicken auf diese Schalfläche wird de*die Nutzer*in aus der Gruppe "Moderatoren" gelöscht und verliert Zugriff auf alle Menus, die nur für Moderator*innen einsehbar sind. Die Kurse, die diese*r Moderator*in bislang betreut hat, sind vorläufig nur noch von Administartor*innen änderbar. \\
	\hline
	Story Points & 6 \\
	\hline
	Entwickler & Tobias Huber \\
	\hline
	Umgesetzt in Iteration & 16\\
	\hline
	Tatsächlicher Aufwand & 7h\\
	\hline
	Velocity & 1.16 h/SP \\
	\hline
	Bemerkung & Beim Klicken auf die Schlatfläche wird in der Datnbank die Verknüpfung zwischen Gruppe "Moderator" und dem Nutzer gelöscht.\\
	\hline
\end{tabularx}
\vspace{20pt}
