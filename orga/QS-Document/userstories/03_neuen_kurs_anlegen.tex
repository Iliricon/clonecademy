\begin{tabularx}{\textwidth}{|p{.2\textwidth}|X|}
	\hline
	ID & 3\\
	\hline
	Benutzerrolle & Moderator\\
	\hline
	Name & Neuen Kurs anlegen\\
	\hline
	Beschreibung & Als Moderator möchte ich einen neuen Kurs anlege können. Dazu gehe ich auf die Kursübersicht und wähle die Schaltfläche ''Add Course'' aus. Ein Editor erlaubt mir, Name und Kategorie des Kurses auszuwählen. Danach kann ich den Kurs speichern und er taucht auf der Übersichtsseite auf. Der Kurs ist erste einmal nur für Moderatoren sichtbar.
\\
	\hline
	Akzeptanzkriterium & Der neu angelegte Kurs mit Name und Kategorie wird persistent gespeichert und ist für Moderatoren einsehbar.\\
	\hline
	Story Points & 3\\
	\hline
	Entwickler & Claas Völcker \& Leonhard Wiedmann\\
	\hline
	Umgesetzt in Iteration & 1\\
	\hline
	Tatsächlicher Aufwand & 3 h\\
	\hline
	Velocity & 1.5 h/SP\\
	\hline
	Bemerkung & Die genaue Funktionsweise des Editors wurde in einem Gespräch mit den Auftraggeber*innen festgelegt.\\
	\hline
\end{tabularx}
\vspace{20pt}
