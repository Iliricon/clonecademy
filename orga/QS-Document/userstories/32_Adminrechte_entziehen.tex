\begin{tabularx}{\textwidth}{|p{.2\textwidth}|X|}
	\hline
	ID & 32 \\
	\hline
	Benutzerrolle & Admin \\
	\hline
	Name & Administrationsrechte entziehen\\
	\hline
	Beschreibung & Als Admin möchte ich einem*einer Administrator*in die Administrationsrechte wieder entziehen können. Wenn ein*e Admin eine*n andere*n Admin in der Administrationsrechte auswählt, gibt es dort einen Button, mit dem man dem*der Admin seine*ihre Adminrechte entziehen kann.  \\
	\hline
	Akzeptanzkriterium & Nach Entzug der Adminrechte kann der*die betreffende Nutzer*in keine administrativen Aufgaben mehr erledigen. Er*sie verliert Zugriff auf alle Bereiche der Seite und Schaltflächen, die nur für Administrator*innen sichtbar sind. Wenn er*sie davor Moderator*in war, behält er*sie die Rechte dieser Gruppe weiterhin. \\
	\hline
	Story Points & 3 \\
	\hline
	Entwickler & Tobias Huber \\
	\hline
	Umgesetzt in Iteration & 16\\
	\hline
	Tatsächlicher Aufwand & 4h\\
	\hline
	Velocity & 1.33 h/SP\\
	\hline
	Bemerkung & Beim Entzug der Administrationsrechte wird die Verknüpfung zwischen dem*der betreffenden Nutzer*in und der Gruppe ''admin'' in der Datenbank gelöscht.\\
	\hline
\end{tabularx}
\vspace{20pt}
