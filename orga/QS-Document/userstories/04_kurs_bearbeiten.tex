\begin{tabularx}{\textwidth}{|p{.2\textwidth}|X|}
	\hline
	ID & 4\\
	\hline
	Benutzerrolle & Moderator*in\\
	\hline
	Name & Kurs bearbeiten\\
	\hline
	Beschreibung & Als Moderator*in möchte ich meine eigenen Kurse bearbeiten können. Dafür soll in der Kursübersicht bei meinen Kursen eine Schaltfläche ''Kurs bearbeiten'' zum Anklicken sein. Nach Anklicken erscheint ein Bearbeitungsfenster, auf dem ich Titel, Zuordnung des Kurses und die einzelnen Module des Kurses bearbeiten kann. Wenn ich die Bearbeitung ohne zu speichern abbreche, wird der Kurs wieder auf seinem ursprünglichen Zustand gesetzt.\\
	\hline
	Akzeptanzkriterium & Der Editor ist für Moderator*innen des Kurses und Administrator*innen nutzbar. Er ist funktional identisch zum ''Kurs anlegen''-Editor. Der bearbeitete Kurs wird korrekt in der Datenbank abgespeichert und auf der Seite aktualisiert. Moderator*innen können nur ihre eigenen Kurse bearbeiten. Wird der Änderungsprozess abgebrochen, ändern sich die Daten in der Datenbank nicht.\\
	\hline
	Story Points & 8 \\
	\hline
	Entwickler & Leonhard Wiedmann \\
	\hline
	Umgesetzt in Iteration & 12\\
	\hline
	Tatsächlicher Aufwand & 8,25 h \\
	\hline
	Velocity & 1.03 h/SP \\
	\hline
	Bemerkung & Der bearbeitete Kurs wird in der Datenbank geändert. Dadurch werden auch abhängige Daten (Module und Frage geändert).\\
	\hline
\end{tabularx}
\vspace{20pt}
