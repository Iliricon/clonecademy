\documentclass[colorback, accentcolor=tud1c, paper=a4]{tudexercise}
\usepackage[ngerman]{babel}
\usepackage[utf8]{inputenc}
\usepackage{multicol}

%opening
\title{Protokoll Nutzerstudie 28.06.2017}
\subtitle{Leonhard Wiedmann}
\subsubtitle{BP Clonecademy}
\author{Leonhard Wiedmann}


\begin{document}

\maketitle
\subsection*{Aufgaben}
\begin{itemize}
	\item Registrierung
	\item Kurs erfolgreich abschließen
	\item Kurs erstellen
\end{itemize}

\subsection*{Probanden}
Es haben 4 Männer und Frauen an der Nutzerstudie teilgenommen.

\subsection*{Probleme}
  Hier sind die Probleme und Anmerkungen der Probanden aufgelistet. Die Zahl in den Klammern zeigt an, wie viele der Probanden es gemerkt haben oder Probleme damit hatten.
  \begin{itemize}
    \item Die Kursliste wird nicht als solche erkannt. Es wurde immer erst auf „Courses“ geklickt. Wurde auch von mehreren Nutzern kritisiert, dass es nicht klar erkennbar ist. (4/4) (ist vermutlich auf den Namen des Testcourses "Testcourse" zurückzuführen)
    \item „+ add module“ ist unverständlich. (3/4)
    \item anstelle „submit“ und „next question“ wäre ein Button besser und übersichtlicher. (3/4) ( aussage der Nutzer )
    \item Wenn eine Frage erneut falsch beantwortet wird, sollte es erkennbar sein, dass es wieder falsch ist. (4/4) ( popup feedback wird hinzugefügt )
    \item Bei ungültiger eingabe des Benutzernamens richtige Fehler anzeigen (2/4) ( fehler in der Form )
    \item Der Button „start Course“ ist verwirrend und sollte nicht extra im Module angezeigt werden. Es wurde der vorschlage gebracht ihn in die Übersichtsliste aufzunehmen. (1/4) ( der button bleibt im Course wird aber von rechts nach links verschoben )
    \item Beim Feedback sollte noch „richtig“ oder „falsch“ dabei stehen. (1/4) ( wird hinzugefügt )
    \item Es sollte erklärt werden, wass ein Kurs und was ein Module ist. ( Punkt für die Auftraggeberbesprechung )
    \item multipLe und nicht multipe
    \item Der Fragentext fällt im Fragen Modul nicht so gut auf wie der learning text und sollte anders hervorgehoben werden (2/4) ( der Style von Fragentext und Learning Text wird getauscht )
    \item anstelle „learning Text“ „module description“ verwenden ( wird übernommen )
   \end{itemize}

	 \subsection*{Sonstiges}
	 	\begin{itemize}
			\item Wegen technischen Problemen wurden keine Videos aufgenommen.
			\item Leider waren nicht mehr Teilnehmer da und somit wurde die Nutzerstudie erfolgreich nach 4 Teilnehmern beendet.
		\end{itemize}
\end{document}
