\documentclass[colorback, accentcolor=tud1c, paper=a4]{tudexercise}
\usepackage[ngerman]{babel}
\usepackage[utf8]{inputenc}
\usepackage{multicol}

%opening
\title{Protokoll Nutzerstudie 06.09.2017}
\subtitle{Leonhard Wiedmann}
\subsubtitle{BP Clonecademy}
\author{Leonhard Wiedmann}


\begin{document}

\maketitle
\subsection*{Aufgaben}
\begin{itemize}
	\item Registrierung
	\item Kurs erfolgreich abschließen
	\item Passwort ändern
\end{itemize}

\subsection*{Probanden}
Es haben 6 Männer und Frauen an der Nutzerstudie teilgenommen.

\subsection*{Probleme}
  Hier sind die Probleme und Anmerkungen der Probanden aufgelistet. Die Zahl in ( ) zeigt an, wie viele der Probanden es gemerkt haben oder Probleme damit hatten. Die erste Zahl sind wie viele es bemängelt haben und die zweite wie viele Probanden gesamt. Die Zahl in [ ] gibt an welche issue id es in Github hat.
  \begin{itemize}
	  \item nach Erfolgreichen ändern des Passwort fehlt ein Feedback (6/6) [40]
		\item ''submit'' gibt kein Feedback, weder für richtig oder falsch (4/6) [38]
		\item unklar wie man sich in dem Fragen Fenster bewegt und Aufgaben abgibt oder weiter kommt (1/6) [43]
		\item Rechtschreibfehler: ''Login with you account'' und im ''register'' beim zweiter Passwort. (Wurde direkt gefixt)
		\item ''User Details'' wurde als Menüpunkt kritisiert, da es nicht sofort klar ist was damit gemeint ist (1/6) \{ Eine Nutzerin welche, ihrer eigenen Aussage nach, wenig Erfahrung mit Computer und Onlineplatformen hat, hat es sofort gefunden. \} (Wird in ''settings'' umbenannt.)
		\item Klarere Unterscheidung von ''dashboard'' zu ''question''. Ein Nutzer hat kritisiert, dass es unklar ist, ob man sich noch auf dem Dashboard oder der Fragenseite ist. (1/6) ( Zu große Designänderung, deshalb wird vorerst nichts gemacht. )
    \item Buttons für Kurse im ''dashboard'' Menü sind nicht ersichtlich. (1/6) (Wenn man auf ''course'' klickt, hätte der Nutzer gerne ein extra Feedback. Kann aber auch dadurch gelöst werden indem mehrere Kurse in einem Reiter stehen.) (Wird vorerst nicht geändert.)
   \end{itemize}

 \subsection*{Sonstiges}
 	\begin{itemize}
		\item Die zugehörigen Videos befinden sich im Ordner
		\item 3 der Teilnehmer waren bereits davor mit der Platform vertraut
	\end{itemize}
\end{document}
