\documentclass[colorback, accentcolor=tud1c, paper=a4]{tudexercise}
\usepackage[ngerman]{babel}
\usepackage[utf8]{inputenc}
\usepackage{multicol}

%opening
\title{Protokol Nutzerstudie 06.09.2017}
\subtitle{Leonhard Wiedmann}
\subsubtitle{BP Clonecademy}
\author{Leonhard Wiedmann}


\begin{document}

\maketitle
\subsection*{Aufgaben}
	\begin{itemize}
	\item Registrierung
  \item Kurs erfolgreich abschließen
  \item Passwort ändern
	\end{itemize}

\subsection*{Probleme}
  Hier sind die Probleme und Anmerkungen der Probanden aufgelistet. Die Zahl in den Klammern zeigt an, wie viele der Probanden es gemerkt haben oder Probleme damit hatten. Die Zahl in [] gibt an welcher issue id es in Github hat.
  \begin{itemize}
    \item \\ nach erfolgreichen ändern des Passwort fehlt ein feedback (6/6) [40]
		\item \\ submit gibt kein feedback, weder für richtig oder Falsch (4/6) [38]
		\item \\ unklar wie man sich in dem Fragen Fenster bewegt und Aufgaben abgibt oder weiter kommt (1/6) [43]
		\item \\ Rechtschreibfehler: "Login with you account" und im register beim zweiter Passwort. Wurde direkt gefixt
		\item \\ "User Details" wurde als Menüpunkt kritisiert, da es nicht sofort klar ist, was damit gemeint ist (1/6) {Eine Nutzerin, welche ihrer eigenen aussage nach wenig Erfahrung mit Computer und Onlineplatformen hat, hat es sofort gefunden. }
		\item \\ Klarere Unterscheidung von Dashboard zu Question. Der Nutzer hat kritisiert, dass es unklar ist, ob man sich noch auf dem Dashboard oder der Fragenseite ist. (1/6)
    \item \\ Buttons für Kurse im Dashboard Menü sind nicht ersichtlich. (Wenn man auf Course klickt, hätte der Nutzer gerne ein extra feedback. Kann aber auch dadurch gelöst werden indem mehrere Kurse in einem Reiter stehen, wodurch es vielleicht ersichtlicher wird.) (1/6) 
   \end{itemize}
\end{document}
