\documentclass[colorback, accentcolor=tud1c, paper=a4]{tudexercise}
\usepackage[ngerman]{babel}
\usepackage[utf8]{inputenc}
\usepackage{multicol}

%opening
\title{Protokol Auftraggebertreffen \today}
\subtitle{Claas A. Völcker}
\subsubtitle{BP Clonecademy}
\author{Claas A. Völcker}


\begin{document}
	
\maketitle

\begin{multicols}{2}
\subsection*{Technische Anmerkungen}
\begin{itemize}
	\item Server muss HTTPS sprechen
	\item Dockernutzer einrichten
\end{itemize}
\subsection*{Inhaltliche Anmerkungen}
\begin{itemize}
	\item Es geht weiter, sobald ein Nutzer die Frage richtig beantwortet hat.
	\begin{itemize}
		\item Fehlerhighlighting an Fragentypus anpassen
	\end{itemize}
	\item Lückentext können wir als kommende Frage aufnehmen
	\item Modulstruktur der Fragen
	\begin{itemize}
		\item Lerngruppen fliegen raus
		\item Nur zwei große Übergruppen: ''iGEM'', ''allg. Biotech''
		\item privat vs. öffentlich fliegt raus, nur sichtbar und nicht sichtbar bleibt (Entwurf und Öffentlich)
		\item Datenbankverknüpfungen per ForeignKey
		\item Fragen sind \emph{immer} in Modulen
	\end{itemize}
	\item Es wird eine Schaltfläche gewünscht: Moderator werden
	\begin{itemize}
		\item Schaltfläche klicken, Mail an Admins, Freischalten oder nicht
	\end{itemize}
	\item iGEM oder Biotech und Sprache wählbar machen (keine einmalige Wahl)
	\begin{itemize}
		\item Sprache ist nur am Anfang auswählbar, damit man nicht mit den angebotenen Kursen durcheinander kommt
	\end{itemize}
	\item Moderator kann einen Kurs nicht selbst freischalten, sondern muss die Freischaltung beim Admin beantragen
	\begin{itemize}
		\item Wenn Kurse einmal freigeschaltet werden,  können sie geändert werden
		\item Wer kann Kurse später bearbeiten?
		\item Kurse brauchen ein Feld: Wer hat angelegt
		\item iGEM-Team berät sich zum Thema Moderations-Whitelist;
	\end{itemize}
	\item Rechteentzug implementieren
	\item Lerntext
	\begin{itemize}
		\item Infotext kommt wo, wird wie eingepflegt?
		\begin{itemize}
			\item Wird im iGEM-Team diskutiert?
		\end{itemize}
	\end{itemize}
\end{itemize}

\subsection*{Qualitätssicherung}
\begin{itemize}
	\item Wie ist Änderbarkeit genau definiert?
	\begin{itemize}
		\item Codemodifizierung einfach, Erweiterungen coden einfach
		\item Wird auf Codeebene gewährleistet
		\begin{itemize}
			\item Kommentare sind wichtig
			\item Übersichtlicher, wartbarer Code
			\item Hooks für Erweiterungen: gutes Vererbungsmodel
		\end{itemize}
	\end{itemize}
\end{itemize}
		
\subsection*{Weiteres Vorgehen}
\begin{itemize}
	\item Wir müssen genauer auf die konkrete Beschreibung in den User Stories achten
	\item Wir haben ein, zwei Kleinigkeiten vergessen
	\item Gestrichene Features verbleiben erst einmal im Code, damit sie später reingenommen werden können
\end{itemize}
\end{multicols}
	
\end{document}