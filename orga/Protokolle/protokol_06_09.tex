\documentclass[colorback, accentcolor=tud1c, paper=a4]{tudexercise}
\usepackage[ngerman]{babel}
\usepackage[utf8]{inputenc}
\usepackage{multicol}

%opening
\title{Protokoll Auftraggebertreffen 9. Juni 2017}
\subtitle{Claas A. Völcker}
\subsubtitle{BP Clonecademy}
\author{Claas A. Völcker}


\begin{document}
	
	\maketitle
	
	\begin{multicols}{2}
		\subsection*{Anmerkung}
		\begin{itemize}
			\item Das Protokoll ist ein Gedächtnisprotokoll, da durch einen Rechnerabsturz das Original verloren gegangen ist.
			\item Eventuelle Unstimmigkeiten bitte so schnell wie möglich melden.
		\end{itemize}
		
		\subsection*{User Stories}
		\begin{itemize}
			\item Die präsentierten Userstories wurden abgenommen (siehe US PDF)
			\item ''Adminrechte freischalten'' (ID 12) wurde in die nächste Iteration verlängert
		\end{itemize}
		
		\subsection*{Funktionalität}
		\begin{itemize}
			\item Man kann Fragen überspringen und später beantworten
			\item Die Texte ändern sich nach Kurs und Modul, müssen also in beiden gespeichert werden
		\end{itemize}
		
		\subsection*{Aufbau der ersten Übersichtsseite}
		\begin{enumerate}
			\item Auswahl der Sprache (Englisch/Deutsch), später nicht mehr änderbar
			\item Auswahl der Kursgruppe: iGEM, Allgemein
		\end{enumerate}
		
		\subsection*{Design}
		\begin{itemize}
			\item Das grundlegende Design, dass von der BP Gruppe präsentiert wurde, ist als Basis gut
			\begin{itemize}
				\item drei Farben, plus weiß und schwarz
				\item Weiß, schwarz und grau sollten als Basisfarben verwendet werden, die anderen beiden als Akzente
				\item (52/56/60), (196/6/51), (214/191/131)
			\end{itemize}
		\end{itemize}
		
		\subsection*{Nutzerrollen}
		Die Namen der Rollen sind vorläufige Arbeitsversionen
		\begin{itemize}
			\item Nutzer: Können Fragen sehen und beantworten, können ihr eignes Profil einsehen
			\item Moderator*innen: Können Kurse neu anlegen und eigene Kurse ändern
			\item Trusted Moderator*innen: Können Kurse anlegen, die nicht mehr explizit freigeschaltet werden müssen, können alle Kurse ändern
			\item Admins: Können Rechte vergeben und entziehen, können Kurse freischalten
			\item Server-Admins: Haben Zugriff auf den Server und das Django Backend
		\end{itemize}
		Die Ordnung über die Gruppen impliziert, dass Nutzer*innen alle Rechte der kleineren Gruppe haben
		\begin{itemize}
			\item Zum Thema Accountlöschung/Blockierung tauscht sich das iGEM Team noch aus.
		\end{itemize}
		
		\subsection*{Weiteres Vorgehen}
		\begin{itemize}
			\item Die kommende Iteration streckt sich über zwei Wochen
			\item Ziel der Iteration ist es, den existierenden Code zu überarbeiten, und Integrationstests durchzuführen, damit ein erster Release auf dem Server möglich ist
			\item Dazu wird das Design umgesetzt und die Datenbank sauber überarbeitet
			\item Alle existierenden Schnittstellen und der Code sollte dokumentiert sein
			\item Der Code wird auf Modularität überprüft
			\item Am 16.06. findet ein eigenes Treffen statt, um technische Fragen zu klären
			\item Als Ziel wird eine Vorstellung der Webseite auf dem iGEM Treffen angestrebt
		\end{itemize}

	\end{multicols}
	
\end{document}

