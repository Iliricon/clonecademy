\documentclass[colorback, accentcolor=tud1c, paper=a4]{tudexercise}
\usepackage[ngerman]{babel}
\usepackage[utf8]{inputenc}
\usepackage{multicol}

%opening
\title{Protokoll Auftraggebertreffen 04.08.}
\subtitle{Claas Völcker}
\subsubtitle{BP Clonecademy}
\author{Claas Völcker}

\begin{document}

\maketitle

\begin{multicols}{2}

\section*{Bericht der letzte Iteration}
\subsection*{User Stories}
\begin{itemize}
	\item bei den Nutzerprofil Änderungen wird die Passwortbestätigung nun korrekt überprüft
	\item Kurs bearbeiten: Bild hochladen funktioniert zur Zeit nur als .jpg außerdem wird das Bild nicht angezeigt

\end{itemize}

\subsection*{Deployment}
\begin{itemize}
	\item Das beste wäre, wenn die Webseite am Montag aus dem Internet erreichbar wäre.
	\item Wir merken an, dass wir keine absolute Sicherheit garantieren können.
	\item Als größte Gefahr wird ein (D)DoS eingeschätzt, der möglicherweise nur die VM abschießt, und nicht die gesamte Serverhardware lahmlegt.
	
\end{itemize}

\subsection*{Wettbewerbsjury}
\begin{itemize}
	\item Beim iGem Wettbewerb werden Juror*innen wenig Zeit haben sich die Webseite anzusehen.
	\item Es wird angestrebt eine leicht zugängliche Demo auch ohne Log-in bereit zu stellen.

\end{itemize}

\section*{Kommende Iteration}
\begin{itemize}
	\item Bilder: Müssen bis Montag angezeigt werden
	\item Infofragetyp: Nur Fragetext, keine Antwortmöglichkeiten, nur auf weiter drücken
	\item Kurs bearbeiten Seite: soll "übersichtlicher" werden: Module und andere Elemente sollen einklappbar sein
	\item Introseite: statische HTML Seite, die Hardgecoded auf dem Server liegt.
	\item Fragen beantworten: Navigation zwischen Fragen in einem Kurs soll möglich sein.

\end{itemize}

\section*{Weitere Planung}
\begin{itemize}
	\item Es wird angestrebt bis Ende August alle relevanten Features einzubringen und nach der zweiten Septemberwoche definitiv keine weiteren großen Features zu implementieren.
\end{itemize}

\end{multicols}
\end{document}
