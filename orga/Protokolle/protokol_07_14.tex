\documentclass[colorback, accentcolor=tud1c, paper=a4]{tudexercise}
\usepackage[ngerman]{babel}
\usepackage[utf8]{inputenc}
\usepackage{multicol}

%opening
\title{Protokol Auftraggebertreffen \today}
\subtitle{Claas A. Völcker}
\subsubtitle{BP Clonecademy}
\author{Claas A. Völcker}


\begin{document}
	
\maketitle

\begin{multicols}{2}

\section{Bericht der letzte Iteration}
\begin{itemize}
	\item Leo präsentiert das Frontend
	\begin{itemize}
		\item Multiple-Choice zu Dritteln macht wenig Sinn, zu viel Platz wird für Feedback verschwendet
		\item Der genaue Aufbau der Frageseite muss mit jedem Fragetypus neu evaluiert werden
		\item Die Bestandteile der Frage könnten noch deutlicher getrennt werden
		\item Vorschlag: Auf der ''Kurs anlegen'' Seite kann man fertig bearbeitete Komponenten wieder einklappen, damit man nicht zu viel Platz verschwendet $\rightarrow$ wird eine User Story
	\end{itemize}
\end{itemize}

\subsection{User Stories}
\begin{itemize}
	\item ''Moderationsrechte freischalten''
	\begin{itemize}
		\item Feedback zur Seite ist noch nicht vorhanden
		\item nicht abgenommen
	\end{itemize}
\end{itemize}

\section{Konkretisierte User Stories}
\begin{itemize}
	\item Wir beschränken uns auf Plasmidkarten
	\begin{itemize}
		\item Es gibt eine kreisförmige Schablone, mit n Feldern
		\item Diese wird vorher implementiert und dann beim Erstellen ausgewählt
	\end{itemize}
	\item Zwischen 3 und 4 Elementen auf einen Kreis
	\item Welche Antwortbausteine gibt es?
	\begin{itemize}
		\item Auch falsche, die gibt der Aufgabensteller ein
		\item Es gibt einen Satz an vorgefertigten Elementen, die man beim Eintragen beschriften kann
	\end{itemize}
\end{itemize}

\section{Zielvereinbarung für die nächste Woche}
\begin{itemize}
	\item Das Refactoring wird beendet
\end{itemize}
\subsection{Neue User Stories}
\begin{itemize}
	\item Drag And Drop (Plasmidkarten) (Moderator/Nutzer)
	\item Nutzerdetails ändern (Nutzer)
	\item Neue Kategorie hinzufügen (Administrator)
	\item Übersicht über das eigene Profil (Nutzer)
\end{itemize}

\section{Weitere Planung}
\begin{itemize}
	\item Das iGEM Team plant einen Release Termin, der passt. 
	\item Dafür werden die Server-Leute gebraucht.
	\item Es wird ein Termin um das erste August Wochenende angestrebt.
\end{itemize}


\end{multicols}
\end{document}
