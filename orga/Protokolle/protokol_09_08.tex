\documentclass[colorback, accentcolor=tud1c, paper=a4]{tudexercise}
\usepackage[ngerman]{babel}
\usepackage[utf8]{inputenc}
\usepackage{multicol}

%opening
\title{Protokoll Auftraggebertreffen 08.09.2017}
\subtitle{Leonhard Wiedmann}
\subsubtitle{BP Clonecademy}
\author{Leonhard Wiedmann}

\begin{document}

\maketitle

\begin{multicols}{2}

\section*{Bericht der letzte Iteration}
\subsection*{User Stories}
\begin{itemize}
	\item US 42 abgenommen
	\begin{itemize}
		\item alle buttons oben außer x entfernen und x rechts alignen
	\end{itemize}
	\item US 43: abgenommen
	\item US 44: abgenommen
	\item US 46: abgenommen
	\begin{itemize}
		\item User kann sich unsichtbar im Ranking setzten
	\end{itemize}
	\item US 31: abgenommen
	\item US 34: abgenommen
	\item US 33: abgenommen
	\begin{itemize}
		\item farbe ändern wird eigene userstory (wenn colorpicker funktioniert)
	\end{itemize}
\end{itemize}

\subsection*{Fragen abschluss}
\begin{itemize}
	\item nach dem Quiz kommt ein Fenster mit Feedback. Hier steht eine liste mit den richtigen und Falsch beantworteten Fragen.
\item Feedback nach Fragen beantworten gibt ein Popup mit Feedback
\end{itemize}


\subsection*{Curse löschen/invisible}
\begin{itemize}
	\item in der Kursübersicht kommt ein Button für visible/invisible (offenes/geschlossenes Auge). Der Button von geschlossen zu offen kann nur von Moderatoren benutzt werden, wobei der von offen zu geschlossen auch vom responsible Mod verwendet werden kann.
\item in der Kursübersicht kommt ein Button für Admins, wodurch man den Kurs löschen kann
\end{itemize}



\end{multicols}
\end{document}
