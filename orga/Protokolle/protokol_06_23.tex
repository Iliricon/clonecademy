\documentclass[colorback, accentcolor=tud1c, paper=a4]{tudexercise}
\usepackage[ngerman]{babel}
\usepackage[utf8]{inputenc}
\usepackage{multicol}

%opening
\title{Protokol Auftraggebertreffen \today}
\subtitle{Claas A. Völcker}
\subsubtitle{BP Clonecademy}
\author{Claas A. Völcker}


\begin{document}
	
\maketitle

\begin{multicols}{2}

\section{Bericht der letzte Iteration}
\subsection{Design}
Wir präsentieren das neue Design

Anmerkungen:
\begin{itemize}
\item Das Menü zur Nutzerauswahl ein Suchfeld haben und kein reines Dropdown sein
\end{itemize}

\subsection{Weitere Anmerkungen}
\begin{itemize}
\item Kurskategorien sollten bei den Kursen auswählbar sein
	\begin{itemize}
	\item z.B. via eines Reiters über der Übersicht im Dashboard
	\end{itemize}
\item Bei der Registrierung sollte die Lernsprache auswählbar sein
\item Der "Get hint" Knopf fehlt noch
	\begin{itemize}
	\item Wird in einer eigenen User Story programmiert
	\end{itemize}
\item Der Kursfortschritt ist noch nicht ganz klar
	\begin{itemize}
	\item Das muss auf der Kursübersicht einsehbar sein (z.B. via Fortschrittsbalken)
	\end{itemize}
\end{itemize}

\section{Vorschlag für neue User Stories:}
\begin{itemize}
\item Nutzerdaten ändern
\item Kurs fortführen
	\begin{itemize}
	\item Knopf "Kurs starten" wird zu "Fortführen"
	\item Man kann einen bereits begonnenen Kurs an jeder Stelle, an der man bereits war, wiederholt starten
	\end{itemize}
\item Moderatorenrechte vergeben
\item Neue Fragen
\item Sprachauswahl bei Registrierung
	\begin{itemize}
	\item Klar machen, dass Englisch Primärsprache ist
	\item Vorschlag: Anzahl der verfügbaren Kurse pro Sprache anzeigen
	\end{itemize}
\end{itemize}

\subsection{Ausgewählt}
\begin{itemize}
\item Moderatorenrechte vergeben
\item Kurs fortführen
\item Sprachauswahl
\end{itemize}

\section{Zielvereinbarung für die nächste Woche}
\begin{itemize}
\item User Stories bearbeiten
\item Refactoring erstmal abschließen
\item Deployment ermöglichen
\end{itemize}

\section{Weitere Planung}
\begin{itemize}
\item Vor einem iGEM Teamtreffen können Nutzerstudien gemacht werden
	\begin{itemize}
	\item Am Anfang eine kleine Gruppe mit viel Beobachtung
	\item Auf Leos Laptop, um Screencapture zu ermöglichen
	\end{itemize}
\end{itemize}
\end{multicols}
\end{document}
