\documentclass[colorback, accentcolor=tud1c, paper=a4]{tudexercise}
\usepackage[ngerman]{babel}
\usepackage[utf8]{inputenc}
\usepackage{multicol}

%opening
\title{Protokoll Auftraggebertreffen \today}
\subtitle{Claas Völcker}
\subsubtitle{BP Clonecademy}
\author{Claas Völcker}

\begin{document}

\maketitle

\begin{multicols}{2}

\section*{Bericht der letzte Iteration}
\subsection*{User Stories}
\begin{itemize}
	\item bei den Nutzerprofil Änderungen wird die Passwortbestätigung nicht richtig überprüft
	\item Kurs bearbeiten: die Fehlermeldung wird nicht korrekt angezeigt
	\item Markdown: angerissen, nicht fertig
	\item Bilder hochladen: abgenommen
	\begin{itemize}
		\item kleine Anmerkung: die Zwischenüberschriften ''Image for the question'' und ''Image for the result'' ist redundant
	\end{itemize}
\end{itemize}

\subsection*{Design}
\begin{itemize}
	\item gewählte Schriftart: Railway (von Google)
	\item finaler Farbcode aus Sample im Seafile
	\begin{itemize}
		\item Achtung: weiß ist eigentlich ein Grauton, bitte beachten
	\end{itemize}
	\item Logo im Seafile: full\_red\_blue\_Main
	\item die Akzentfarbe Blau wird den bisherigen Sandton ersetzten, die Buttons werden Grau, das BP Team setzt ein paar Kombinationsmöglichkeiten um
	\item in einem eigenen Treffen kann das Design dann weiter verfeinert werden
\end{itemize}

\section*{Kommende Iteration}
\begin{itemize}
	\item kommende Woche soll das Deployment stattfinden
	\begin{itemize}
		\item 7. August, 14:00 Uhr, im Schreibraum des iGEM Team, Claas schließt sich mit Markus kurz
	\end{itemize}
	\item Design-Treffen: Dienstag, 14:00 Uhr, an der Biologie Seminarraum Botanik, bis dahin werden die Farben eingearbeitet
	\item Skalierung der hochgeladenen Bilder muss bedacht werden
\end{itemize}

\section*{Weitere Planung}
\begin{itemize}
	\item YouTube Embedding wurde angesprochen
	\begin{itemize}
		\item wurde vom BP Team als machbar eingestuft
	\end{itemize}
\end{itemize}

\end{multicols}
\end{document}
